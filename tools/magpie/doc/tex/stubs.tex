\chapter{Client and service stubs}
\label{chapter.stubs}

This chapter shows how client and service stubs generated by Magpie may be integrated into code. All examples in this section work with the IDL file {\tt twoifaces.idl}, shown in full in figure \ref{figure.twoifaces.idl}.

Examples in this chapter are L4-specific.

\begin{figure}
\begin{verbatim}
/*
 * Two rather-unrelated interfaces.
 */

[uuid(22)]
interface iguana_pd
{
    int create_memsection(in int inputarg, inout int result);
};

[uuid(23)]
interface rng
{
    void randomise(in int seed);
    int rand();
};
\end{verbatim}
\caption{{\tt twoifaces.idl}}
\label{figure.twoifaces.idl}
\end{figure}

\section{Using client stubs}

Use of Magpie stubs as a client is straightforward. Magpie flattens the interfaces by folding the interface name into the function name using an underscore, so functions for a specific interface are called {\tt interfacename\_functionname}. For {\tt twoifaces.idl}, the function call is shown in Figure \ref{figure.stub.client.example}. Note that in order to call the service, you need a reference to the thread implementing that service (of type {\tt iguana\_pd} in this case). This reference is simply an L4 thread ID (type {\tt L4\_ThreadId\_t}). Magpie does not provide a way for client threads to obtain the thread IDs of services, so this must be done using some other mechanism.

\begin{figure}
\begin{verbatim}
#include "simple-client.h"

[...]

void
test_func(iguana_pd service)
{
    int inputarg, outputarg;
    int result;
    CORBA_Environment env;

    inputarg = 37; /* Any input here. */
    
    result = iguana_pd_create_memsection(service, inputarg, &outputarg, &env);

    [...]
}
\end{verbatim}
\caption{Using Magpie client stubs}
\label{figure.stub.client.example}
\end{figure}

\section{Using service stubs}

To use service stubs, you must write a {\it service loop}. Magpie can provide a template for a standard service loop which you can use as a basis for extension with the {\tt -t} option.

\begin{verbatim}
$ ./magpie.py -t twoifaces.idl -p generic
\end{verbatim}

It is important to remember that the service template may be different depending on the platform, thus it is important to specify the correct platform when generating the template.

An example template (the result of running the above command line) is shown in figure \ref{stubs.twoifaces.template}. The important parts of the template are discussed below.

\subsection{Templated function definitions}


\begin{figure}
\begin{verbatim}
/* 
 * This is an automatically-generated file.
 * Source file  : twoifaces.idl
 * Output file  : -
 * Platform     : V4 Generic
 * Mapping      : CORBA C
 *
 * Generated by multi_idl 6, March 4 2005
*/

#include "twoifaces_service.h"


IDL4_INLINE int iguana_pd_create_memsection_implementation(CORBA_Object _caller, int inputarg, int *result, idl4_server_environment *_env)
{
    int retval;

    /* Implementation of iguana_pd::create_memsection */

    return retval;
}

/* Link the name of the function above with the name defined in the default vtable. */
IDL4_PUBLISH_IGUANA_PD_CREATE_MEMSECTION(iguana_pd_create_memsection_implementation);

/* Use the default dispatch table defined in the service header. */
void * iguana_pd_vtable[IGUANA_PD_DEFAULT_VTABLE_SIZE]
        = IGUANA_PD_DEFAULT_VTABLE;

IDL4_INLINE void rng_randomise_implementation(CORBA_Object _caller, int seed, idl4_server_environment *_env)
{

    /* Implementation of rng::randomise */

}

/* Link the name of the function above with the name defined in the default vtable. */
IDL4_PUBLISH_RNG_RANDOMISE(rng_randomise_implementation);

IDL4_INLINE int rng_rand_implementation(CORBA_Object _caller, idl4_server_environment *_env)
{
    int retval;

    /* Implementation of rng::rand */

    return retval;
}

/* Link the name of the function above with the name defined in the default vtable. */
IDL4_PUBLISH_RNG_RAND(rng_rand_implementation);

/* Use the default dispatch table defined in the service header. */
void * rng_vtable[RNG_DEFAULT_VTABLE_SIZE]
        = RNG_DEFAULT_VTABLE;
void server(void)
{
    L4_ThreadId_t partner;
    L4_MsgTag_t msgtag;
    idl4_msgbuf_t msgbuf;
    long cnt;

    while (1) {
        partner = L4_nilthread; /* Our initial reply is to the nilthread. */
        msgtag.raw = 0;
        cnt = 0;

        while (1) {
            idl4_reply_and_wait(&partner, &msgtag, &msgbuf, &cnt);
            if (idl4_is_error(&msgtag)) {
                /* FIXME: Add your error handler here. */
                printf("server: error sending IPC reply\n");
                break; /* Reset thread ID to nilthread & try again */
            }
            switch(magpie_get_interface_id(&msgbuf)) {


                case 22:
                    idl4_process_request(&partner, &msgtag, &msgbuf, &cnt,
                            iguana_pd_vtable
                            [idl4_get_function_id(&msgtag) && IGUANA_PD_FID_MASK]);
                    break;


                case 23:
                    idl4_process_request(&partner, &msgtag, &msgbuf, &cnt,
                            rng_vtable
                            [idl4_get_function_id(&msgtag) && RNG_FID_MASK]);
                    break;

            }
        }
    }
}
\end{verbatim}
\caption{twoifaces\_template.c}
\label{stubs.twoifaces.template}
\end{figure}

