\chapter{Usage}

Magpie is invoked from the command line. The simplest command line consists of no options and a single IDL filename. Given this command-line, a client stub header file for the generic V4 architecture is produced and written to {\tt stdout}.

Magpie can emulate the IDL4 command-line interface using the {\tt magpidl4.py} program. This is currently Magpie's recommended interface if you are not planning on extending Magpie, and is described below\footnote{If you do plan on extending Magpie, you may prefer to use Magpie's generic interface. See Chapter \ref{chapter:templating} for more information.}.

\begin{verbatim}
$ ./magpidl4.py test/input/simple.idl4

/* 
 * This is an automatically-generated file.
 * Source file  : test/input/simple.idl4
 * Output file  : -
 * Platform     : V4 Generic
 * Mapping      : CORBA C
\end{verbatim}
{\it[ etc ]}

\section{Magpie command-line options}
Common options for the IDL4-compatible interface, {\tt magpidl4.py}, are described below.

\begin{description}
\item{{\tt -h}} Specify an output filename. If this option is not specified, or the filename supplied is `-', output is written to {\tt stdout}.
\item{{\tt -c, --client-stubs}} Generate client stubs. This is the default. Client stubs, service stubs, and the template service loop are explained in Chapter \ref{chapter.stubs}.
\item{{\tt -s, --service-stubs}} Generate service stubs.
\item{{\tt -t}} Generate a template service loop.
\item{{\tt -p, --platform=}} Specify the platform. The default platform is {\tt generic}, which produces output compatible with IDL4's generic mode. Other supported platforms are {\tt generic\_biguuid} and {\tt arm\_biguuid}. Platform types are explained below.
\item{{\tt -I}} Specify include paths. Include paths are necessary when the specified IDL file imports C or C++ headers. They are used by Magpie to find the imported header, but are also passed to the preprocessor.
\item{{\tt --with-cpp=}} Specify the C preprocessor. The preprocessor is invoked to parse imported C or C++ header files. The default is `cpp'. 
\end{description}

\section{Supported platforms}
Magpie supports a number of different {\em platforms}. In the IDL4 interface generator, the underlying hardware of the system for which generated interfaces are to be built is specified using the platform parameter. Specifying a specific supported hardware platform to IDL4 produces output optimised (and only usable on) that platform, comprising a mixture of C and assembly-language code. Alternatively, the {\tt generic} platform can be specified. This platform, supported by all hardware architectures, produces pure C output.

IDL4 separates the platform, which specifies the hardware interface, from the {\em interface}, which describes the API used to communicate with the kernel. For example, the {\tt v4} interface corresponds to the L4 V4 API. Magpie provides a more general way to customise to a particular interface, and so the separate concepts of platform and interface are merged into Magpie {\em targets}. Magpie preserves these concepts for the IDL4 emulation interface.

Magpie provides three platforms through its IDl4-specific interface. Each platform changes the output of Magpie in hardware-specific or calling-convention-specific ways. The supported platforms are described below.

\subsection{Platform {\tt generic}}
The {\tt generic} platform uses the L4 generic functions to implement IPC (such as {\tt L4\_Ipc()}). This mode is compatible with stub code generated by IDL4. The L4 V4 API provides up to 64 machine-word-sized {\em message registers} for transmission of information from one thread to another during IPC; the generic platform uses message register 0 to store the interface number and function ID of the called interface and marshalls and unmarshalls data into and out of the remaining message registers automatically. Because a single message register is used for interface number and function number, this interface allows up to 1024 unique interfaces per thread, and 64 functions per interface.

\subsection{Platform {\tt generic\_biguuid}}
The {\tt generic\_biguuid} platform uses the same L4 generic functions, but uses an extra message register to store interface IDs --- message register 1. This allows for a much larger number of interfaces per thread (up to the archictecture's word size) at the cost of one less message register for client data.

\subsection{Platform {\tt arm\_biguuid}}
The {\tt arm\_biguuid} platform is functionally the same as generic\_biguuid, but sets up and calls L4's IPC functions directly using inline assembly language for the ARM platform. This provides for faster IPC, because it takes advantage of an ARM platform optimisation that allocates four hardware registers for the first four message registers, meaning that these registers do not have to be copied during IPC.

