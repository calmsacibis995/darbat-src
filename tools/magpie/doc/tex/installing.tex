\chapter{Installation}
This chapter covers installation and testing of Magpie. Most of the instructions assume a Unix-alike system, such as Linux, but Windows is also briefly discussed.

\section{Obtaining Magpie}
Released versions of Magpie may be found on Magpie's official homepage at the Distributed Systems group of UNSW: \url{http://www.disy.cse.unsw.edu.au/Software/Magpie} \footnote{If you are a member of the DiSy group at UNSW, development versions of Magpie may be checked out from the DiSy Arch repository using the repository name ``magpie--mainline''}.

\section{Prerequisites}
You will need Python 2.3 or higher, and an installation of the {\tt dparser} Python library.

\subsection{Installation on Linux-based systems}
Python comes standard on most Linux systems --- {\tt python -V} will tell you the version. If your system does not have Python installed, it should be available in your distribution. As a last resort, Python source code and RPM packages are available on \url{http://www.python.org}.

To install Dparser, check out a copy from \url{http://dparser.sourceforge.net/} and follow the installation instructions. Note that you will need Python development headers in order to compile Dparser's Python extension. For Debian Linux, this means you will need the {\tt python-dev} package.

\subsection{Installation on Windows}
Install Python using the official Windows installer at \url{http://www.python.org}. Install Dparser using the Windows binary at \url{http://staff.washington.edu/sabbey/py\_dparser/}.

\section{Testing}
When Dparser is successfully installed, you may test Magpie by running a self-test.

\begin{verbatim}
$ ./magpidl4.py test/input/simple.idl4
\end{verbatim}

If Magpie is working correctly, it will write client interface stubs to stdout.

To perform a full set of self-tests, run the test suite. The test suite may take several minutes to execute.

\begin{verbatim}
$ python test/test.py
\end{verbatim}

If you are planning on using Magpie with the Iguana system, you may run an extended set of self-tests using Iguana's header files by passing appropriate flags to the tester. Run {\tt python test/test.py --help} for more information.

\section{Installing Magpie}
Magpie may be run in-place, or you may install it. To install magpie, run the installer:

\begin{verbatim}
$ python install.py
\end{verbatim}

By default, Magpie is installed to {\tt /usr/local/magpie}. Run the installer with the {\tt --help} option for instructions on changing this location. You do not need to run the installer as root, provided this installation directory is writable by the installer.

\subsection{Make Magpie run faster}
The first time Magpie is run, it creates two sets of parser tables --- one for its IDL parser, and one for its C++ parser. It stores these tables in the directory in which the {\tt magpie.py} or {\tt magpidl4.py} resides. If these tables cannot be written, they will be recreated each time, resulting in slower operation. Therefore, it is a good idea to run Magpie's self-tests after installation.

\section{Uninstalling Magpie}
Magpie may be uninstalled by removing the installation directory or by running the installer. By default, the installer takes care not to remove files if they have changed since installation.

\begin{verbatim}
$ python install.py --uninstall
\end{verbatim}
